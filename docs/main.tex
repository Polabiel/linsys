\documentclass[11pt]{article}

% line spacing
\linespread{1.5}

% paragraph spacing
\setlength{\parindent}{1.0cm}
\setlength{\parskip}{0.2cm}

% packages
\usepackage{amsfonts}
\usepackage{amsthm}
\usepackage{amsmath}
\usepackage[brazil]{babel}
\usepackage{booktabs}
\usepackage{caption}
\usepackage{float}
\usepackage[T1]{fontenc}
\usepackage{graphicx}
\usepackage{hyperref}
\usepackage[utf8]{inputenc}
\usepackage{lipsum}
\usepackage{microtype}
\usepackage{nicefrac}
\usepackage{rotating}
\usepackage{template}
\usepackage{url}
\usepackage{xcolor}

% definitions
\theoremstyle{definition} \newtheorem{definition}{Definição}

% examples
\theoremstyle{definition} \newtheorem{example}{Exemplo}

% colors
\definecolor{javared}{rgb}{0.6,0,0} % for strings
\definecolor{javagreen}{rgb}{0.25,0.5,0.35} % comments
\definecolor{javapurple}{rgb}{0.5,0,0.35} % keywords
\definecolor{javadocblue}{rgb}{0.25,0.35,0.75} % javadoc

% packages settings
\graphicspath{{./images/}}

\hypersetup
{
    colorlinks = true,
    linkcolor = blue,
    citecolor = blue,
    urlcolor = blue,
    bookmarksdepth = 4
}

\title{Projeto Sistemas de equações lineares} 
\author{Gabriel Oliveira dos santos \\ \texttt{cc23600@g.unicamp.br} \And Julia Vitória Santos Bacellar \\ \texttt{cc23526@g.unicamp.br}}

\begin{document}

\maketitle

\section{Introdução}

{Nós estudantes (Gabriel e Julia) do \textit{\href{https://cotuca.unicamp.br/}{Colégio Técnico de Campinas}}, fomos introduzidos a resolver um problema de Sistema de equações lineares utilizado a linguagem \href{https://www.python.org/}{Python} com a biblioteca auxiliar \href{https://scipy.org/}{Scipy}.}
{Sistemas de equações lineares são fundamentais em várias áreas da ciência, tecnologia e engenharia, aparecendo frequentemente em problemas que requerem a determinação de valores que satisfaçam múltiplas condições lineares simultâneas. Para resolver esses sistemas de forma eficiente, o \href{https://pt.wikipedia.org/wiki/Algoritmo_simplex}{Algoritmo Simplex} se destaca como uma técnica robusta de otimização linear. Este método encontra a solução ótima de um problema linearmente restrito de maneira iterativa.}

\subsection{Objetivos}

{E por sua vez, esse relatório tem como objetivo apresentar uma aplicação do Algoritmo Simplex na resolução de sistemas de equações lineares utilizando um software, que foi desenvolvido pelos alunos. bem como analisar os resultados obtidos para diferentes configurações de instâncias.}

\section{Experimentos computacionais}

{Todos os experimentos computacionais foram realizados em um PC com sistema operacional Windows, equipado com um processador \href{https://www.intel.com.br/content/www/br/pt/products/sku/52209/intel-core-i52500-processor-6m-cache-up-to-3-70-ghz/specifications.html}{Intel(R) Core(TM) i5-2500 CPU @ 3.30GHz 3.70 GHz} e 12 GB de memória RAM. O Algoritmo Simplex foi implementado em Python utilizando a biblioteca SciPy.}

\subsection{Implementação}

{Como foi dito no \href{https://cdn.discordapp.com/attachments/990092066694524980/1251955505941839892/projeto2.pdf?ex=667075ec&is=666f246c&hm=c3939979dac83efa757c6360efbd0db8b65ccf560cd1109bc99dcf5f84f42808&}{enuciado}, Para resolver os sistemas de equações lineares, utilizamos a linguagem de programação Python junto com a biblioteca SciPy, que oferece funcionalidades robustas para otimização e resolução de problemas lineares. Abaixo, descrevemos como implementamos o algoritmo Simplex para resolver diferentes problemas de otimização linear:}
\begin{enumerate}
    \item Definição da Função Objetivo: Cada problema requer a definição de uma função objetivo que queremos minimizar ou maximizar. Por exemplo, a função objetivo pode ser algo como "\(\min{ 5x_1 + x_2}\)" ou "\(\max 2x_1 - 3x_2\)".
    \item Restrições: As restrições de cada problema são especificadas em forma de desigualdades (por exemplo, \(2x_1 + x_2 \geq 6\)) e igualdades (por exemplo, \(x_1 + x_2 = 4\)). Essas restrições são transformadas em uma matriz de coeficientes e um vetor de termos independentes.
    \item Limites das Variáveis: Definimos os limites para as variáveis envolvidas no problema. Normalmente, estas variáveis são restritas a valores não-negativos.
    \item Resolução com Simplex: Utilizamos a função linprog da biblioteca SciPy para resolver os problemas de otimização linear utilizando o método Simplex. A função retorna a solução ótima, o valor da função objetivo e informações sobre o processo de otimização.
    \item Análise dos Resultados: Os resultados obtidos são analisados e apresentados para cada problema, incluindo os valores das variáveis, o valor da função objetivo e mensagens de status do algoritmo.
\end{enumerate}

\section{Algoritmo Simplex}

{O Algoritmo Simplex é uma abordagem iterativa para a resolução de problemas de programação linear. Ele inicia a partir de uma solução viável básica e, em cada iteração, se desloca para uma solução adjacente que melhora o valor da função objetivo, até que uma solução ótima seja alcançada. No contexto de sistemas de equações lineares, o Simplex pode ser empregado para encontrar soluções de problemas formulados como programas lineares. O algoritmo é eficiente e pode ser aplicado a problemas de grande escala.} {
O algoritmo Simplex é um método para resolver problemas de programação linear. Ele é usado para encontrar o valor máximo ou mínimo de uma função objetivo, sujeito a um conjunto de restrições. No código fornecido, o algoritmo Simplex é usado para resolver vários problemas de otimização.\\O código usa a função linprog da biblioteca scipy.optimize para implementar o algoritmo Simplex. A função linprog recebe os seguintes parâmetros:
\begin{itemize}
    \item c: Uma lista de coeficientes da função objetivo.
    \item A\_ub: Uma matriz de coeficientes para as restrições de desigualdade.
    \item b\_ub: Um vetor de termos independentes para as restrições de desigualdade.
    \item bounds: Uma lista de tuplas representando os limites inferiores e superiores para cada variável.
    \item method: O método a ser usado para resolver o problema de otimização. Neste caso, é usado o método 'highs', que é uma implementação do algoritmo Simplex.
\end{itemize}
Cada função \textbf{\textit{problem\_X}} define um problema de otimização específico, configurando os coeficientes da função objetivo, as restrições e os limites das variáveis de acordo com o problema. Em seguida, a função linprog é chamada para resolver o problema.

A função \textbf{\textit{print\_result}} é usada para exibir os resultados da otimização. Ela recebe o resultado da função linprog, o nome do problema e a solução esperada, e imprime a solução encontrada e a solução esperada.

Por exemplo, no \textit{problem\_1}, a função objetivo é \(5x_1 + x_2\) e as restrições são \(2x_1 + x_2 \geq 6\), \(x_1 + x_2 \geq 4\), \(x_1 + 5x_2 \geq 10\) e \(x_1, x_2 \geq 0\). A função linprog é chamada com os coeficientes e restrições correspondentes, e a solução encontrada é impressa.

Por favor, note que a função linprog minimiza a função objetivo. Portanto, quando o problema é maximizar a função objetivo, os coeficientes da função objetivo são negativos.
}


\section{Resultados}

{No resultado é possivel visualizar dentro do Github as instances com todas as declarações de erro, Os resultados dos experimentos realizados são apresentados na Tabela \ref{tab:resultados}. Esta tabela mostra o gap, a função objetivo, o valor capturado e a equação realizada para cada instância testada.

A seguir, apresentamos os resultados dos problemas resolvidos utilizando o algoritmo Simplex. Para cada problema, listamos os valores das variáveis \(x\), a função objetivo, o status da otimização, o número de iterações e a solução ótima esperada.

\begin{table}[H]
    \centering
    \caption{Resultados dos Problemas}
    \begin{tabular}{cccc}
        \toprule
        \textbf{Problema} & \textbf{Variáveis} & \textbf{Função Objetivo} & \textbf{Iterações} \\
        \midrule
        1 & \( x_1 = 0.0, x_2 = 6.0 \) & 6.0 & 0 \\
        2 & \( x_1 = 4.0, x_2 = 0.0 \) & -8.0 & 1 \\
        3 & \( x_1 = 1.0, x_2 = 1.0, x_3 = 0.0 \) & -56.0 & 0 \\
        4 & \( x_1 = 400.0, x_2 = 0.0, x_3 = 0.0, x_4 = 0.0 \) & 0.0 & 0 \\
        5 & \( x_1 = 0.0, x_2 = 1.0, x_3 = 11.0 \) & -33.0 & 0 \\
        6 & Não resolvido & - & - \\
        7 & \( x_1 = 8.3188, x_2 = 0.0 \) & -74.87 & 2 \\
        \bottomrule
    \end{tabular}
    \label{tab:resultados}
\end{table}

\subsection{Análise dos Resultados}

A tabela acima apresenta uma visão geral dos resultados obtidos para cada problema. A seguir, detalhamos cada problema:

\begin{itemize}
    \item \textbf{Problema 1:} A solução ótima obtida foi \( x_1 = 0.0 \) e \( x_2 = 6.0 \), com uma função objetivo de 6.0. O status do algoritmo foi 0, indicando uma solução ótima encontrada sem problemas, com 0 iterações.
    \item \textbf{Problema 2:} A solução ótima obtida foi \( x_1 = 4.0 \) e \( x_2 = 0.0 \), com uma função objetivo de -8.0. O status foi 0, com 1 iteração.
    \item \textbf{Problema 3:} A solução ótima obtida foi \( x_1 = 1.0 \), \( x_2 = 1.0 \) e \( x_3 = 0.0 \), com uma função objetivo de -56.0. O status foi 0, com 0 iterações.
    \item \textbf{Problema 4:} A solução ótima obtida foi \( x_1 = 400.0 \), \( x_2 = 0.0 \), \( x_3 = 0.0 \) e \( x_4 = 0.0 \), com uma função objetivo de 0.0. O status foi 0, com 0 iterações.
    \item \textbf{Problema 5:} A solução ótima obtida foi \( x_1 = 0.0 \), \( x_2 = 1.0 \) e \( x_3 = 11.0 \), com uma função objetivo de -33.0. O status foi 0, com 0 iterações. Houve um gap de 3 unidades na função objetivo esperada.
    \item \textbf{Problema 6:} Este problema não foi resolvido pois não foi especificado no enunciado.
    \item \textbf{Problema 7:} A solução ótima obtida foi \( x_1 = 8.3188 \) e \( x_2 = 0.0 \), com uma função objetivo de -74.87. O status foi 0, com 2 iterações.
\end{itemize}

}

\section{Conclusões}

{O Algoritmo Simplex demonstrou-se eficaz na resolução de sistemas de equações lineares, proporcionando soluções ótimas de forma eficiente. Os resultados obtidos para a instância de teste validam a precisão e a eficiência do método. Para problemas de maior escala, o Simplex continua sendo uma ferramenta poderosa, embora técnicas adicionais possam ser necessárias para melhorar o desempenho computacional.

Em conclusão, o Algoritmo Simplex é uma abordagem robusta para resolver sistemas de equações lineares, sendo amplamente aplicável em diversas áreas que requerem soluções ótimas para problemas lineares.}

\end{document}