\documentclass[11pt]{article}

% line spacing
\linespread{1.5}

% paragraph spacing
\setlength{\parindent}{1.0cm}
\setlength{\parskip}{0.2cm}

% packages
\usepackage{amsfonts}
\usepackage{amsthm}
\usepackage{amsmath}
\usepackage[brazil]{babel}
\usepackage{booktabs}
\usepackage{caption}
\usepackage{float}
\usepackage[T1]{fontenc}
\usepackage{graphicx}
\usepackage{hyperref}
\usepackage[utf8]{inputenc}
\usepackage{lipsum}
\usepackage{microtype}
\usepackage{nicefrac}
\usepackage{rotating}
\usepackage{template}
\usepackage{url}
\usepackage{xcolor}

% definitions
\theoremstyle{definition} \newtheorem{definition}{Definição}

% examples
\theoremstyle{definition} \newtheorem{example}{Exemplo}

% colors
\definecolor{javared}{rgb}{0.6,0,0} % for strings
\definecolor{javagreen}{rgb}{0.25,0.5,0.35} % comments
\definecolor{javapurple}{rgb}{0.5,0,0.35} % keywords
\definecolor{javadocblue}{rgb}{0.25,0.35,0.75} % javadoc

% packages settings
\graphicspath{{./images/}}

\hypersetup
{
    colorlinks = true,
    linkcolor = blue,
    citecolor = blue,
    urlcolor = blue,
    bookmarksdepth = 4
}

\title{Projeto $<\#>$} 
\author{Gabriel Oliveira dos santos \\ \texttt{cc23600@g.unicamp.br} \And Julia Bacellar \\ \texttt{cc<ra>@g.unicamp.br}}

\begin{document}

\maketitle

\section{Introdução}

{O `LINSYS` é um trabalho em equipe desenvolvido por alunos da disciplina de Topicos de inteligencia artificial, ministrada pelo professor Guilherme Macedo, no curso de Desenvolvimento de Sistemas no Colégio técnico de Campinas - UNICAMP. Com base no conteúdo abordado em sala de aula, o projeto tem como objetivo a implementação de um sistema de equações lineares, utilizando a linguagem de programação Python. O sistema deve ser capaz de resolver sistemas de equações lineares, calcular a inversa de uma matriz, calcular o determinante de uma matriz, calcular a transposta de uma matriz, calcular o produto de duas matrizes, calcular a matriz adjunta e calcular a matriz identidade.}

\subsection{Objetivos}

\section{Experimentos computacionais}

\subsection{Implementação}

\section{Métodos}

\section{Resultados}

\section{Conclusões}

\end{document}